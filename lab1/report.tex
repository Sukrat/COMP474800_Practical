\documentclass[12pt]{article}    
    \usepackage[utf8]{inputenc}
    \usepackage[margin=3cm]{geometry}
    
    \title{\vspace{-3.0cm}COMP47480 Practical 1: Extreme Programming Exercise}
    \author{Sukrat Kashyap (14200092)}
    \date{30 January 2018}
    \setlength{\parindent}{0cm}
    \setlength{\parskip}{0.7em}
    
\begin{document}

\maketitle

\section{Work Done}

We were group of 5 people. Our task was to build a fridge using the Extreme programming methodology. So, at the start we divided our group into 2 consumers and 3 developers. I was one of the developers in the Iteration 1. Consumers were then supposed to create stories telling about the features they want in the fridge. These stories by the consumers were given to us to give the expected time to finish the feature. We developers then estimated the time required for the stories and gave the cards back to the consumers. Some stories were not clear like ``It has two sided door''. So we asked the consumers whether they want the doors for both the compartments or one or they want one big two sided door for both the compartments. After clarifying this detail we drew a small picture of the same to make it clear. After giving the cards back we gave our work time of 3 minutes for the iteration. The cards were given back to us and we three developers one by one took stories and implemented them.

In similar fashion, we did the second iteration. Only this time the roles were switched. So, then we wrote the features on top of the fridge that was already being created. The developers clarified some details such as how big TV they want. The estimates of the stories and their total work effort were given. We then picked all the stories as the estimated time of the stories were completely fitting their effort time. They then took the stories one by one and build the device on the previous implemented fridge.

After, the 2 iterations we retrospected on the whole process. The customers were happy with the product. Developers had few difficulties implementing as some of the request were bizzare but doable. Queries were solved satisfactorily but most of the decision were given to developers as what they might seem fit.

\section{Reflections}

Extreme Programming is one of many software development methodologies which aims to improve software quality and responsiveness to changing customer requirements. It is a type of agile software development which has both incremental and iterative properties. These properties tackles the varying project requirements (One of the major problems in software development). 

In the practical, we used Extreme programming methodology to develop our fridge. Only 2 iterations were performed. This gave us an insight to the development process. The two properties namely incremental and iterative differences were also clear. In simple terms, incremental was adding new features to our fridge and iteration was adding these new features repetitively. It  also helped us to understand the two different mindset (developers and consumers). EP felt like a program which decoupled two entities (business and development) in software development and created a link between the two. Following the first Agile manifesto i.e. Individuals and interaction over process and tools. While developing fridge there was no guideline of documenting or even thinking and caring about future needs but it strongly with held the idea of testing each feature before saying its complete, covering the second agile manifesto i.e. Working software over comprehensive documentation. The developers collaborated with the consumer to understand the feature of the fridge before estimating their time and also communicating when implementing the same; regards to the third agile manifesto i.e. Customer collaboration over contract negotiation. Few modification of the original feature was done to accomodate new feature in second iteration claims the fourth agile manifesto which is Responding to change over following a plan.

One of the other advantages that I felt was that the process was much simpler to understand and implement. Unlike other Agile practices such as Scrum. Scrum seems a bit complicated as it contains various roles such as Product owner, scrum master, team members etc. and contains multiple artifacts such as product backlog, sprint backlog, burn down charts, Task board etc. Whereas, when it comes to EP the process is much simpler having only 2 roles that is the developers and consumers with the artifacts being the story cards. Scrum seems to have fine detail about each roles and seems to be a bit less flexible. Whereas the EP abstracts the management of each roles internal management to themselves. Giving flexibility to the developers and consumers to manage their internal affairs. EP seems to look like an collaboration and interaction process between the developers and consumers. While saying this, EP does have guidelines which the 2 entities must follow for better throughput. For e.g. 
\begin{itemize}
    \item For developers
    \begin{itemize}
        \item Pair programming
        \item Testing first approach also know as (TDD Test driven development)
    \end{itemize}
    \item For consumers
    \begin{itemize}
        \item Stories must be a feature that the consumers with no implementation detail
        \item Stories chosen to be developed must not be more than the promised effort number given by the developers
    \end{itemize}
\end{itemize}

The incremental property of Agile practices in general gives the business side of the company to evaluate their product in the market after every release which helps reducing the cost. Unlike the Waterfall process where the defect is usually found in later stages. Making the process revisit analysis, design and implementation part. Whereas in agile process, the defects are found early, giving the opportunity to the business side to re-evaluate their product. Which usually involves in spin-off of the already released product and adding new features to counter attack the problems. This helps increasing the business value of the product and reducing the overall cost.

At the end, it felt Extreme programming was a nice way to development. But still, it lacked more stronger and finer guidelines and rules when it comes to handling bigger team. We were a group of 5. So, it seemed pretty easy to follow and implement the ideology. But it seems to be naive when it comes to bigger teams on much bigger project. Since its iterative, it seems harder to create software which has deadlines. One of the problems that I realise in Agile in general is that when the developers start working on an iteration the stories are locked and not allowed by the consumers to change. So, a major defect or bug in the system which needs immediate care during the iteration is harder to include. Also, most of bugs or defect are minor and doesn't need much changes but some of them are major and it becomes harder to estimate the time to fix. The most problematic part of the EP is the estimation of stories, in EP the developers estimation is somewhat vague and is not always true. EP says to finish the story in the time being estimated. But in reality it doesn't happen. Estimated time are mostly underestimations and it is due to different developers have different speed in development. Even though other agile methodologies sometimes uses a vague numbering system for estimation called story points. If the stories are not finished then they are moved to the next iteration/sprint. A team's number of average story points achieved in a sprint and their current estimation is used as a measure to estimate the total work effort of the team. This seems a better approach than asking the developers their effort time for an iteration.

\end{document}
