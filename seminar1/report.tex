\documentclass[12pt]{article}    
    \usepackage[utf8]{inputenc}
    \usepackage[margin=3cm]{geometry}
    
    \title{\vspace{-3.0cm}COMP47480 Seminar 1: IBM}
    \author{Sukrat Kashyap (14200092)}
    \date{20 February 2018}
    \setlength{\parindent}{0cm}
    \setlength{\parskip}{0.7em}
    
\begin{document}

\maketitle

\section{More than Software}

\begin{itemize}
    \item Having the right offering
    \item Getting to the market
    \item Selling it
    \item Operating it
    \item Supporting it
    \item Evolving it
\end{itemize}

\section{More than Engineering}

Many other crucial professions/disciplines are involved in managing the Software

\begin{itemize}
    \item Management
    \item Project Management
    \item Business
    \item designers
    \item Test Operations
    \item Supporting
    \item Sales and Marketing
    \item Pricing
    \item Legal
\end{itemize}

\section{Lifecycle of Project}

IBM uses Agile methodology in their lifecycle.

\begin{itemize}
    \item Vision
    \item Plan
    \item Develop
    \item Deliver
    \item Operate
\end{itemize}

\subsection{Vision}

IBM uses a Design thinking pattern which is process for innnovating and delivering fast. It adds certain practices namely hills, playbacks, sponsor users.

\begin{itemize}
    \item Hills: it is expressed as an aspirational end state for users that is motivated by market understanding. It defines a mission and scope of a release. No more than 3 major release are recommended with a technical foundation.
    \item Playbacks: Moving forward requires a lot of feedbacks and that's where playbacks come into play. All design and deelopment work is iterative.
    \item Sponsor users: these are the people who are selected from real or intended user group. By working with sponsor users. It allows for better design experiences for real target users, rather than imagines needs.
\end{itemize}

\subsection{Plan}

Planing involves number of steps like mapping down story boards.

Their planning is similar to Agile planning where we have Epics -> Features -> Plan items -> Stories.

Stories are the leaf items that the Project team work on. Test, documentation and deployment is a part of every story and must be followed by the team members.

Planning is known as Playback 0. They use IBM Rational Team Concert (Jazz) tool to manage all aspects of their work, such as iteration, release planning, change management, defect tracking, source control, and build automation.

\subsection{Develop}

Development in IBM uses Eclipse as IDE, Ration Software architec (UML), RTC for source control, Tomcat as webserver, DB2 for database, JUnit for testing, Selenium for web browser testing, CheckStyle for static code analysis, Sonar Qube for continuous inspection of code quality like code smells and security vulnerabilities.

They use Jenkins/Build forge for continuous testing. Scripting is done using gradle and artifacts for managing software artifacts and metadata.

All the software is deployed for testing on WebSphere and DB2. Testing is an important part of development and a story or a feature is given clearance for deployment once when following checks are done.

Functional Verification
System Verification
Business Verification
Peer code reviews

This process is Playback N

\subsection{Deliver}

The projects follows continuous delivery and deployment. Releases are done every 2 X 2 weeks.

\subsection{Operate}

This involves multiple disciplines such as Deploy, Monitor, Support. Support organisation has multiple heirarchy e.g. (L1/L2/L3)

\section{Manage Evolution}

Release are done monthly
Separate stream for parallel development
Merging Streams
Check points
Legal clearance

\end{document}
