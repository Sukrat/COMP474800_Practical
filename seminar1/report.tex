\documentclass[10pt]{article}    
    \usepackage[utf8]{inputenc}
    \usepackage[margin=3cm]{geometry}
    
    \title{\vspace{-3.0cm}COMP47480 Seminar 1: IBM}
    \author{Sukrat Kashyap (14200092)}
    \date{20 February 2018}
    \setlength{\parindent}{0cm}
    \setlength{\parskip}{0.7em}
    
\begin{document}

\maketitle
\section{Speaker}

Paddy Fagan, Chief Architect at IBM Watson Health. He is working across SAAS and on premise solutions in healthcare and government. He has worked to bring together the customer team's requirements and the Product Development Organization to architect approaches and solutions that are the best fit for customers.

\section{More than Software}

He said that a successful piece of software doesn't mean a successful project. It was much more than only the software. It involved lot of different parts which had to work together to create successful projects. Some of those parts are as follows:

\begin{itemize}
    \item Having the right offering: Giving something that the customers you are targeting actually need and will benefit
    \item Getting to the market
    \item Selling it: selling it at a price that the end users can afford it
    \item Operating it
    \item Supporting it: supporting your customers whenever fault occurs and also helping them to get used to eat
    \item Evolving it: while providing the services, the service should evolve according to the customer's needs
\end{itemize}

\section{More than Engineering}

Many other crucial professions/disciplines are involved in managing the Software

\begin{itemize}
    \item Engineers: know what cannot be done
    \item Project Management: knows how to organise the project and how it should be worked towards
    \item Business: These people know what should be done
    \item Designers: knows how it should look like
    \item Test Operations
    \item Supporting: needed for helping the end customer with their queries
    \item Sales and Marketing: knows how to sell it
    \item Legal: department knows the laws and regulation that it should comply with
\end{itemize}

\section{Lifecycle of Project}

IBM uses Agile methodology in their lifecycle which involves Vision, Plan, Develop, Deliver and Operate.

\subsection{Vision}

IBM uses a Design thinking pattern which is process for innnovating and delivering fast. It adds certain practices namely hills, playbacks, sponsor users.

\begin{itemize}
    \item Hills: it is expressed as an aspirational end state for users that is motivated by market understanding. It defines a mission and scope of a release. No more than 3 major release are recommended with a technical foundation.
    \item Playbacks: Moving forward requires a lot of feedbacks and that's where playbacks come into play. All design and deelopment work is iterative.
    \item Sponsor users: these are the people who are selected from real or intended user group. By working with sponsor users. It allows for better design experiences for real target users, rather than imagines needs.
\end{itemize}

\subsection{Plan}

Planing involves number of steps like mapping down story boards.

Their planning is similar to Agile planning where we have Epics (themes), Features (user identifiable features), Plan items (development iterations) and Stories (development sprints).

Stories are the leaf items that the Project team work on. Test, documentation and deployment is a part of every story and must be followed by the team members.

Planning is known as Playback 0. They use IBM Rational Team Concert (Jazz) tool to manage all aspects of their work, such as iteration, release planning, change management, defect tracking, source control, and build automation.

\subsection{Develop}

Development in IBM uses Eclipse as IDE, Ration Software architect (UML), RTC for source control, Tomcat as webserver, DB2 for database, JUnit for testing, Selenium for web browser testing, CheckStyle for static code analysis, Sonar Qube for continuous inspection of code quality like code smells and security vulnerabilities.

They use Jenkins/Build forge for continuous testing. Scripting is done using gradle and artifacts for managing software artifacts and metadata.

All the software is deployed for testing on WebSphere and DB2. Testing is an important part of development and a story or a feature is given clearance for deployment once when following checks are done.

\begin{itemize}
    \item Functional Verification is equivalent program verification
    \item System Verification
    \item Business Verification
    \item Peer code reviews
\end{itemize}

This process is Playback N

\subsection{Deliver}

The projects follows continuous delivery and deployment. Releases are done every 2 X 2 weeks.

\subsection{Operate}

This involves multiple disciplines such as Deploy, Monitor, Support. Support organisation has multiple heirarchy e.g. (L1/L2/L3)

\section{Manage Evolution}

Release are done monthly
Separate stream for parallel development
Merging Streams
Check points
Legal clearance

\section{Q \& A}

Some question that were asked at the end of seminar were:

\begin{enumerate}
    \item Software project has 3 main characterstics namely Performance, Features and Stability. When and how one would prioritize them? \\
    Answer: Priority depends on the stage and type of project. Usually in the start Feature is the most important to showcase in the market what new you are bringing, then it shifts to stability when more and more people start using it and at last when you have competitors as well the priority shifts to performance.
    \item How to update/maintain the support team? \\
    Answer: Youtube videos, documentation or a group demo is done weeks before the release.
    \item Pair programming? \\
    Answer: Pair programming is something that is up to the team. But usually teams donot prefer pair programming.
    \item UML diagram in IBM? \\
    Answer: IBM uses tools to create the UML diagrams from the code like Rational Software architect.
    \item Development methodology used in IBM? \\
    Answer: Very agile methodologies are used and Software like IBM Rational Team Concert.

\end{enumerate}

\end{document}
