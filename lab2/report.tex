\documentclass[12pt]{article}    
    \usepackage[utf8]{inputenc}
    \usepackage[margin=3cm]{geometry}
    
    \title{\vspace{-3.0cm}COMP47480 Practical 2: UML Exercise}
    \author{Sukrat Kashyap (14200092)}
    \date{13 February 2018}
    \setlength{\parindent}{0cm}
    \setlength{\parskip}{0.7em}
    
\begin{document}

\maketitle

\section{Work Done}

In this practical, we were given to create Use-case model, Domain model and Interaction model of a library system. The library had books and journals which could be borrowed by the members of the library (Students and Staff). There a restriction on the number of items each member could borrow. Journals could only be borrowed by the Staff members. There were two types of loan for the books namely short-term (4 hours) and long-term (4 weeks). We had to represent the system in 3 different model representation. We first created use-case model for the library system which was done by finding different actors (student, staff, and member) and use-cases such as borrow journal, check availability and common use-cases such as borrow book, return, and display. After the use-case model, we picked out nouns suitable for making classes like Student, Member, Staff, Copy, Book, Journal and created links between them. We then created an interaction diagram which showed us scenarios of communication between the classes and had to add a new booking system to our domain model. We then were told to add a new requirement of notifying the members if they had the book or journal for too long. To which we added a booking system which notified the member when the booked in all of our models.

\section{Reflections}



\end{document}
