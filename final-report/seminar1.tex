\subsection{Speaker}

Paddy Fagan, Chief Architect at IBM Watson Health. He is working on SAAS and on-premise solutions in healthcare and government. He has worked to bring together the customer team's requirements and the Product Development Organization to architect approaches and solutions that are the best fit for customers.

\subsection{Report}

A successful piece of software doesn't mean a successful project. It was much more than only the software. It involved a lot of different parts which had to work together to create successful projects. This was the highlight of the seminar. It is quite counter-intuitive as I especially thought that company is super dependent on its software and better software corresponds to better sales which in turn helps in the success of the company. But after the seminar, the highlight became clearer. For IBM some of the major parts of software development were about:

\begin{itemize}
    \item Having the right offering: Giving something that the customers you are targeting actually need and will benefit
    \item Getting to the market
    \item Selling it: selling it at a price that the end users can afford it
    \item Operating it
    \item Supporting it: supporting your customers whenever a fault occurs and also helping them to get used to eat
    \item Evolving it: while providing the services, the service should evolve according to the customer's needs
\end{itemize}

The above points before the talk felt like is not necessary to focus on as these will happen on its own as long as the software is at its best. But no, that was not the case, even if the software is super intelligent and performant but is not something that the customers need. It will not benefit anyone. If it is not at the right price or has no proper support for the customers it won't grow. These points made sense when it was deeply explained by the speaker.

\subsubsection{More than Engineering}

Since it was required to focus on many different areas when building a software. It meant that we need not only engineers but other professions. All the other crucial professions/disciplines are involved in managing the Software collectively and missing of even one of the professions will lead to a not so successful project/software. The following major disciplines were talked about in the seminar and their role.

\begin{itemize}
    \item Engineers: know what cannot be done
    \item Project Management: knows how to organise the project and how it should be worked towards
    \item Business: These people know what should be done
    \item Designers: knows how it should look like
    \item Test Operations
    \item Supporting: needed for helping the end customer with their queries
    \item Sales and Marketing: knows how to sell it
    \item Legal: department knows the laws and regulation that it should comply with
\end{itemize}

\subsubsection{Lifecycle of Project}

After explaining the importance of the other professions involved in the project. The speaker explained the lifecycle of the project which is used in IBM. Which I believe is very crucial as it is quite easy to work in an already developing phase of a project. But what about the initial phase? How would you start the project? According to IBM, the project lifecycle involves Vision, Plan, Develop and deliver which are done iteratively and then Operate.

\begin{itemize}
    \item Vision: IBM uses a Design thinking pattern which is the process of innovating and delivering fast. It adds certain practices namely hills, playbacks, sponsor users. The aim of this stage is to produce a 'to be' vision of how the users of the software should work. They use Post Its, docs, presentation to represent the aims. 
    \item Plan: This involves mapping down the steps toward the vision. These steps are similar to agile planning where we have Epics (themes), Features (user identifiable features), Plan items (development iterations) and Stories (development sprints).
    \item Develop: In this stage actual development of the software is done.
    \item Deliver: This involves deployment of the software.
    \item Operate: Multiple disciplines such as Deploy, Monitor, Support comes into play.
\end{itemize}

In most of my college career, every assignment that I did was usually in a code and fix methodology where I directly jumped into the code. After seeing and understanding this process. I felt that this is a much cleaner approach. It at first seemed like it will take more time and resources to follow these steps. But to my surprise, it made the environment much easier to work in. This is because most of the problems related to design and use-case were fixed already in the vision and planning stage. Leaving us to development phase where the idea of what to be done was clear and created a smooth path to be followed on.

\subsubsection{Develop and Deliver}

IBM is such a large multi-national technology company that it requires proper mechanism or methodology to develop and deliver software. IBM uses Agile methodology throughout their company for development. I feel sticking to one methodology actually helps in communicating and managing numerous different departments and the company as a whole. Agile methodology these days is used by a number of companies to track the progress and manage the evolution of the software. I learnt many benefits in agile in comparison with other processes such as code and fix, waterfall, Kanban etc. This was because of the nature of the development of software. It was nice to see the actual implementation of the methodology in such a multi-national company. 

In practical 1, we actually role-played extreme programming development in a small team. This methodology seemed to naive but the whole environment seemed much nicer to develop in. It was understandable that this cannot be employed in larger teams or bigger projects. Even though XP is an agile development, the one mostly used by IBM was the scrum. Scrum is a very sophisticated and in-depth process containing much more detailed regulations to be followed. Now comparing to what we learnt about waterfall model of development. This model actually didn't feel like it would suit IBM as much as Scrum agile because sequential design and customers ever-changing needs make it unbearable.

For developing, IBM mostly used Open-source software due to its huge community support. They use Eclipse as IDE, Ration Software architect (UML), RTC for source control, Tomcat as the web server, DB2 for the database, JUnit for testing, Selenium for web browser testing, CheckStyle for static code analysis, Sonar Qube for continuous inspection of code quality like code smells and security vulnerabilities. Due to its huge code base, code reviewing and Sonar Qube were the main inspectors for code smells, performance and design issues. For such large code base, documentation is important as well. For this IBM uses Youtube, group demo and other various methods to document the infrastructure of the application as well as for the support team to understand the usage.

They use Jenkins/Build forge for continuous testing. Scripting is done using Gradle and artefacts for managing software artefacts and metadata.

All the software is deployed for testing on WebSphere and DB2. Testing is an important part of the development and a story or a feature is given clearance for deployment once when following checks are done.

\begin{itemize}
    \item Functional Verification is equivalent program verification
    \item System Verification
    \item Business Verification
    \item Peer code reviews
\end{itemize}

Testing is very important in software development. IBM uses a lot of resources and has a number checks and filter before any new feature is been release into the production. They perform unit tests, integration test and User acceptance test before putting it on the release. These checks and verification listed above actually help in identifying a number of bugs early on. This proper planning sometimes can cause a lot of difficulty as a major bug in the system would require the story to go back to the iteration and making it harder to accommodate. 

Agile, since is iterative would require setting a priority in mind when assigning stories and Software project usually have 3 main characteristics namely Performance, Features and Stability. According to the speaker, the priority was determined on the stage of the project. Usually in the start Feature is the most important to showcase in the market what new you are bringing, then it shifts to stability when more and more people start using it and at last when you have competitors as well the priority shifts to performance.

In the end, it was quite interesting to see that IBM actually cared about each and every small important things into account like refactoring, code smell analysis etc. There organised and formal way of approaching development keeping in mind the whole ecosystem of the company consisting of different departments is also quite applaudable.