\subsection{Speaker}

Richard Sheehan (Production Engineer) and Mike Elkin (Site Reliability Engineer) at Facebook.

\subsection{Report}

The speakers started with a statement which said that ``Failure is not an option but it is inevitable''. They meant to focus on the fact that Failure is a part of Software Development. It is something we definitely do not want but at the same time it is inevitable and cannot be deferred. The main idea behind was to accept this nature of Development and the team should be able to learn from them and make sure it never happens again. 

They showcased the vast Facebook's data centre all over the world. Each datacenter was built in order to manage massive traffic coming from not only Facebook but from Instagram, Whatsapp etc. They also showed the switches that they build for the data centres which they chose over purchasing from the vendors because of the cost. Each data centre was supposed to have a massive infrastructure to handle large traffic from the network. Now with such big infrastructure possibility of error is vast. Identifying and resolving them is another big challenge. They gave us an example of how one of the errors that occurred shut down the entire Facebook data centre. Performance is also important and with such large infrastructure, it becomes harder to optimize and increase performance. They showcased their software development methodology by giving an example of improving their performance of data transfer between networks to users.

Their methodology does not include agile planning or software development. It consists of abstract project development method which is as follows:
\begin{enumerate}
      \item Investigation
      \item Analysis
      \item Project Retrospect
      \begin{itemize}
            \item Set clear project goals
            \item If you cannot measure it, it did not happen
            \item Get the right stakeholders
            \item Requirements gathering
            \item Career progression is not linear
            \item Sunk cost fallacy (if not working move on)
      \end{itemize}
\end{enumerate}

The overall structure is quite similar to IBM's Vision, Plan and Develop. The words involved are quite different and so is their idea behind it. Usually, companies focus on software development as building a machine whereas Facebook's development is in terms of numbers if a project tends to increase the performance they ask themselves by how much or if it helps to solve some problem they ask themselves how many people would it help solve the problem. The number of people working in a team in Facebook is usually around 6-10. Most of the companies in my experience follow certain methodology like Agile, Waterfall etc. Whereas, Facebook leaves everything to the team. Facebook has more of an ``ownership'' way of development where they give the project to a team and then the project is owned by the team in a sense that they can decide whatever it seems fit.

Facebook has over billion users in each of their platforms so when it comes to errors, they need robust methodology to tackle the errors occurred and a plan so that it doesn't ever happen again. Their immediate step can be stated as:

\begin{itemize}
      \item Reboot servers
      \item Disable automation
      \item Check rest of the site's status
      \item look for config changes or deployment
\end{itemize}

These steps are simple enough to tackle errors occurred followed by small hack or fix to keep the site running for the time being. After this software engineers gather together in ``War room''. Where they try to tackle the problem keeping in mind the following guidelines:

\begin{itemize}
      \item Try to keep the rest of the site running
      \item Reproduce failures on a small scale
      \item Look for code/config changes
      \item Dig into and analyse data
\end{itemize}

Now, this is the immediate measures taken by the engineers at Facebook. But since we need something to fix the problem such so that it does not happen again. For this they try to:

\begin{itemize}
      \item Focus on the root cause (Deep dive and Not blame)
      \item Find how to prevent recurrence
      \item Do not rely on best intentions
      \item Always remember humans are desperately unreliable      
\end{itemize}

These are usually the guidelines and rules for tackling, handling and solving the errors that occur in the system. They showed with real-life errors that actually occurred on Facebook and how they followed the above steps and guidelines to fix the memory leak and overloading of traffic on servers problems.

To prevent failures unit testing, integration testing and load testing are one of the ways. But testing would not catch all the failure nodes and one can't test an entire distributed system at scale. Hence, they try to:
\begin{itemize}
      \item Roll out stuff slowly
      \item Monitor KPIs
      \item Fail fast than timeouts
      \item Defensive servers
\end{itemize}

One can see here that most of the points that are being showcased by the speakers are not hard and fast rules. They are also not in any book of best practices. These rules and regulations seem more like discoveries that they made and the method they found out to suit their needs using their past experiences. Learning and moving forward is one of the Facebook ideals and it can be seen in their methodology.

Since contribution, changes, code and bugs are all proportional to each other. Hence, small changes help them to figure out what broke when. The error tackling process is very different from the normal agile developments where most of the errors are caught before the development pipeline and if not immediate fix is created but proper fix is done in the later stages of the cycle of development whereas in Facebook they do not of have cycle of development, they release as they build and if error occurs they rollback and fix and then keep developing.

The code rolling out is in small chunks but every day. This is what surprised me! They not only roll out every day but several times a day. Whereas in agile, waterfall and kanban processes, rolling out of new code is every 2-4 weeks. Every team in a company usually follows a proper agile planning whereas in Facebook it is on ownership basis. Testing and code reviews are the norm these days and are followed by most of the companies even Facebook. But Facebook has a really large source code spread over several modules, computers, data centres. Hence running the test every time they deploy is almost impossible. Hence to tackle this they have tools that usually run against what they develop and only those test and their dependencies are run. Saving a lot of time in the process.

I feel such varied development process is due to the sheer number of computing and data. If Facebook had to use agile planning it would fail because rolling out such big change is risky. This risk comes because of its a distributed systems. Distribution is not only between computers but whole data centres. Rolling out to such large data centres is another impossible task. With such large code base testing, each of them would take ages. Finding error is another difficult task for them because of this sheer velocity and amount. Hence, they do something call reactive programming where they automate most of the tasks and wait for the signal from the system if something is wrong. Facebook also have automated tools for detecting software vulnerabilities, code smell etc. Facebook uses their own whereas IBM and other companies use SonarQube.

It is very surprising to see such different approach of development compared to IBM (which is discussed later). Their approach from development to testing is totally different, unique and interesting. Their approach is more of a summary of their solutions of their past mistakes. They not only understand that there is no best way to develop a particular software and every software has its own need. They also understand and realize the need for automating almost everything and using reactive programming where it's more about notification than polling for new information. Their understanding of these issues allows them to be so flexible.
