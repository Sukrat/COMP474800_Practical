\subsection{Speaker}

Roy Phillips, Chief Technical Officer at FoodCloud.

\subsection{Report}

The speaker started off with explaining about FoodCloud. FoodCloud has deals with the various grocery stores. When grocery stores cannot sell perfectly good food. They upload the description of the food using their in-store scanner or their smartphone app. FoodCloud has many links with the local charity. They let the local charity know about the unsold food that the grocery stores couldn't sell. The charity can respond to the notification by accepting and collecting it.

They not only have client and charity app, they have text notification service, a website and have various integration with their clients. The toughest part in FoodCloud is the integration of their platform with their clients i.e Tesco etc. Most of the companies usually deal with providing services that they usually make from scratch and run on their own servers or in the cloud. Integration is usually one of their tasks but mostly they are the providers providing the services. FoodCloud, on the other hand, acts as a middleman where they create a bridge between the grocery stores and the charity. This requires software to be very versatile and flexible to be able to handle integration with ease.

FoodCloud uses Scala as the main programming language which addresses the criticisms of Java. Scala also uses Java VM. This is due to the robustness and cross-platform Java VM that the FoodCloud chose to use. FoodCloud has to integrate with multiple clients hence they do require cross-platform scalable tools to be used.

In regards to the running of the application, Facebook and IBM have their own big data centres as they are such large companies. Whereas since FoodClous is a growing non-profit business, they need a scalable solution. For this reason, FoodCloud chose Heroku for deploying the apps for its ease of deploying and scalable plus they also are expandable based on needs meaning they only have to pay for what they use.

Tracking of the progress of the software development was done using JIRA. They employed simple backlog mechanism where they pick the most important tasks and work on it. Tracking the task in the backlog. They followed Kanban and Agile methodology. The speaker did not believe in estimating the time required to finish the task. As in software development, it's very hard to estimate how much time it is gonna take to complete a task. Wasting time on estimation is hence unnecessary. This ideology is quite valid in real life. Spikes are very common in software development. Spike is essentially when trying to finish a task, the developers find that this task actually requires major work and more time than the estimation. These spikes tend to obstruct the agile process. In my internship, there was a number of occasions that I estimated certain stories with time. And most of the times, either I finished the story too quickly or most of the times too late. This is because, until one starts coding, one cannot estimate the time required to finish as one does not know the number of changes that will be required to achieve the goal. Hence removing the time factor from the development helps get rid of the delays and makes the development more flexible. Hence the speaker followed a methodology where his team took the maximum of 4 stories and worked on them. Being said that, I believe not estimating makes it harder for the business side of the company to fall behind as their work is more date oriented. It also becomes harder to predict the future plans of the company.

The speaker strongly followed the SOLID principles. SOLID principles are the design principles in OOPs. FoodCloud used technology and designs such as SRP, Akka Streams, RabbitMQ for messaging etc. Testing was given a lot of importance by the speaker. I feel the technologies chosen by the speaker are indeed quite robust and scalable but tests are still important. To do so the author, still created the test for database calling directly even though it was slow as no memory-based solution was there for PostgreSQL. This shows the importance of testing. It might let your productivity down but in long term, it actually increases the overall efficiency. The hosting was on Heroku which does make sense as the want minimize their cost. They followed continuous integration and development which allowed them to iteratively working on the software, testing it on the server as well as production as they go along. 

Even though the speaker did not use anything for Use-case and software architecture creation. He used a simple whiteboard to do so. The important to understand is that planning and documentation is an important part of the development. Using of tools is not necessary as it is up to the developer.

I feel that the process and the tools are well suited and chosen for the development of FoodCloud. It is interesting to see different methodologies and process used by various different companies. The number of projects, the size of the project and the type of the project drive the type and methodology chosen. It is interesting that the speaker takes a different approach quite similar to Facebook. Rather than following strict methodology they tend to focus on simple backlog like implementation. This is due to the sheer size of users and the size of the company. Another reason for choosing so that I believe is that FoodCloud requires them to interact with various developers of the client the company is working with to integrate their system.
