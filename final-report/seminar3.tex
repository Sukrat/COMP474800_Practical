
\section{Speaker}

Roy Phillips, Chief Technical Officer at FoodCloud.

\section{Report}

The speaker started off with explaining about FoodCloud. FoodCloud has deals with the various grocery stores. When grocery stores cannot sell perfectly good food. They upload the description of the food using their in-store scanner or their smartphone app. FoodCloud has many links with the local charity. They let the local charity know about the unsold food that the grocery stores couldn't sell. The charity can respond to the notification by accepting and collecting it.

They not only have client and charity app, they have text notification service, a website and have various integration with their clients. The toughest part in FoodCloud is the integration of their platform with their clients i.e Tesco etc. We can see that IBM and Facebook mostly deal with providing services that they usually make from scratch and run on their own servers. Integration is usually one of their tasks but mostly they are the providers providing the services. FoodCloud, on the other hand, acts as a middleman where they create a bridge between the grocery stores and the charity. This requires software to be very versatile and flexible to be able to handle integration with ease.

Facebook did not seem to give much care to the software tools or platform they used. IBM mostly used Java for creating programs. FoodCloud uses Scala which addresses the criticisms of Java. Scala also uses Java VM. This is due to the robustness and cross-platform Java that the FoodCloud chose to use. FoodCloud has to integrate with multiple clients hence they do require cross-platform scalable tools to be used.

In regards to the running of the application, Facebook and IBM have their own big data centres as they are such large companies. Whereas since FoodClous is a growing non-profit business, they need a scalable solution. For this reason, FoodCloud chose Heroku for deploying the apps for its ease of deploying and scalable plus they also are expandable based on needs meaning they only have to pay for what they use.

Tracking of the progress of the software development was done using JIRA. They employed simple backlog mechanism where they pick the most important tasks and work on it. Tracking the task in the backlog. They followed Kanban and Agile methodology. The speaker did not believe in estimating the time required to finish the task. As in software development, it's very hard to estimate how much time it is gonna take to complete a task. Wasting time on estimation is hence unnecessary. This ideology is quite valid in real life. Spikes are very common in software development. Spike is essentially when trying to finish a task, the developers find that this task actually requires major work and more time than the estimation. These spikes tend to obstruct the agile process. Hence the speaker followed a methodology where his team took the maximum of 4 stories and worked on them. 

The speaker strongly followed the SOLID principles. SOLID principles are the design principles in OOPs. FoodCloud used technology and designs such as SRP, Akka Streams, RabbitMQ for messaging etc. Testing was given a lot of importance by the speaker. I feel the technologies chosen by the speaker are indeed quite robust and scalable. The hosting on Heroku does make sense as the want minimize their cost. They follow continuous integration and development. 

I feel that the process and the tools are well suited and chosen for the development of FoodCloud. It is interesting to see different methodologies and process used by various different companies. The number of projects, the size of the project and the type of the project drive the type and methodology chosen. IBM used one process to manage different teams namely agile methodology. Whereas Facebook did not care much about the process and gave the developers the freedom to choose their methodology. It is interesting that Roy takes a different approach quite similar to Facebook. Rather than following strict methodology they tend to focus on simple backlog like implementation. This is due to the sheer size of users and the size of the company. Another reason for choosing so that I believe is that FoodCloud requires them to interact with various developers of the client the company is working with to integrate their system.
