This section explains the various tools and methodologies used by the three different companies that came to give a talk on their development process. The three companies are IBM, Facebook and FoodCloud. Also, these companies and compared and contrasted in the last section.

\section{IBM seminar}
\subsection{Speaker}

Paddy Fagan, Chief Architect at IBM Watson Health. He is working across SAAS and on premise solutions in healthcare and government. He has worked to bring together the customer team's requirements and the Product Development Organization to architect approaches and solutions that are the best fit for customers.

\subsection{Report}

A successful piece of software doesn't mean a successful project. It was much more than only the software. It involved lot of different parts which had to work together to create successful projects. This was the highlight of the seminar. It is quite counter intuitive as I specially thought that company is super dependent on its software and better software corresponds to better sales which in turn helps in succes of the company. But after the seminar the highlight became more clearer. For IBM some of the major parts of software development was about:

\begin{itemize}
    \item Having the right offering: Giving something that the customers you are targeting actually need and will benefit
    \item Getting to the market
    \item Selling it: selling it at a price that the end users can afford it
    \item Operating it
    \item Supporting it: supporting your customers whenever fault occurs and also helping them to get used to eat
    \item Evolving it: while providing the services, the service should evolve according to the customer's needs
\end{itemize}

The above points before the talk felt like is not necessary to focus on as these will happen on its own as long as the software is at its best. But no, that was not the case, even if the software is super intelligent and performant but is not something that the customers need. It will not benifit anyone. If it is not at right price or has no proper support for the customers it won't grow. These points made sense when it was deeply explained by the speaker.

\subsubsection{More than Engineering}

Since it was required to focus on many different areas when building a software. It meant that we need not only engineers but other profession. All the other crucial professions/disciplines are involved in managing the Software collectively and missing of even one of the professions will lead to a not so successful project/software. The following major disciplines were talked about in the seminar and their role.

\begin{itemize}
    \item Engineers: know what cannot be done
    \item Project Management: knows how to organise the project and how it should be worked towards
    \item Business: These people know what should be done
    \item Designers: knows how it should look like
    \item Test Operations
    \item Supporting: needed for helping the end customer with their queries
    \item Sales and Marketing: knows how to sell it
    \item Legal: department knows the laws and regulation that it should comply with
\end{itemize}

\subsubsection{Lifecycle of Project}

After explaining the importance of the other professions involved in the project. The speaker explained the lifecycle of the project which is used in IBM. Which I believe is very crucial as it is quite easy to work in a already developing phase of a project. But what about the initial phase? How would you start the project. According to IBM, the project lifecycle involves Vision, Plan, Develop and deliver which are done iteratively and then Operate.

\begin{itemize}
    \item Vision: IBM uses a Design thinking pattern which is process for innnovating and delivering fast. It adds certain practices namely hills, playbacks, sponsor users. The aim of this stage is to produce a 'to be' vision of how the users of the software should work. They use Post Its, docs, presentation to represent the aims. 
    \item Plan: This involves mapping down the steps toward the vision. These steps are similar to agile planning where we have Epics (themes), Features (user identifiable features), Plan items (development iterations) and Stories (development sprints).
    \item Develop: In this stage actual development of the software is done.
    \item Deliver: This involves deployment of the software.
    \item Operate: Multiple disciplines such as Deploy, Monitor, Support comes into play.
\end{itemize}

In most of my college career, every assignment that i did was usually in a code and fix methodology where i directly jumped into the code. After seeing and understanding this process. I felt that this is a much cleaner approach. It at first seemed like it will take more time and resources to follow these steps. But to my surprise, it made the environment much easier to work in. This is because most of the problems related to design and use-case was fixed already in the vision and planning stage. Leaving us to development phase where the idea of what to be done was clear and created a smooth path to be followed on.

\subsubsection{Develop and Deliver}

IBM is such a large multi-national technology company that it requires proper mechanism or methodology to develop and deliver software. IBM uses Agile methodology throught out their company for development. I feel sticking to one methodology actually helps in communicating and managing numerous different departments and the company as a whole. Agile methodology these days is used by a number of companies to track the progress and manage the evolution of the software. I learnt many benifits in agile in comparison with other processes such as code and fix, waterfall, kanban etc. This was because of the nature of development of software. It was nice to see the actual implemention of the methodology in such a multi-national company. 

In practical 1, we actually role-played extreme programming development in a small team. This methodology seemed to naive but the whole environment seemed much nicer to develop in. It was understandable that this cannot be employed in larger teams or bigger projects. Even though XP is an agile development, the one mostly used by IBM was the scrum. Scrum is a very sophisticated and in depth process containing much more detailed regulations to be followed. Now comparing to what we learnt about waterfall model of development. This model actually didnot feel like it would suit IBM as much as Scrum agile because sequential design and customers ever changing needs makes it unbearable.

For developing, IBM mostly used Open-source software due to its huge community support. They use Eclipse as IDE, Ration Software architect (UML), RTC for source control, Tomcat as webserver, DB2 for database, JUnit for testing, Selenium for web browser testing, CheckStyle for static code analysis, Sonar Qube for continuous inspection of code quality like code smells and security vulnerabilities. Due to its huge code base, code reviewing and Sonar Qube were the main inspectors for code smells, performance and design issues. For such large code base, documentation is important as well. For this IBM uses Youtube, group demo and other various methods to document the infrastructure of the application as well as for the support team to understand the usage.

They use Jenkins/Build forge for continuous testing. Scripting is done using gradle and artifacts for managing software artifacts and metadata.

All the software is deployed for testing on WebSphere and DB2. Testing is an important part of development and a story or a feature is given clearance for deployment once when following checks are done.

\begin{itemize}
    \item Functional Verification is equivalent program verification
    \item System Verification
    \item Business Verification
    \item Peer code reviews
\end{itemize}

Testing is very important in software development. IBM uses a lot of resources and have a number checks and filter before any new feature is been release into the production. They perform unit tests, integration test and User acceptance test before putting it onto the release. These checks and verification listed above actually help in identifying a number of bugs early on. This proper planning sometimes can cause a lot of difficulty as a major bug in the system would require the story to go back to the iteration and making it harder to accomodate. 

Agile, since is iterative would require setting a priority in mind when assingning stories and Software project usually have 3 main characterstics namely Performance, Features and Stability. According to the speaker the priority was determined on the stage of the project. Usually in the start Feature is the most important to showcase in the market what new you are bringing, then it shifts to stability when more and more people start using it and at last when you have competitors as well the priority shifts to performance.

\section{Facebook seminar}
\subsection{Speaker}

Richard Sheehan (Production Engineer) and Mike Elkin (Site Reliability Engineer) at Facebook.

\subsection{Report}

The speakers started with a statement which said that ``Failure is not an option but it is inevitable''. They meant to focus on the fact that Failure is a part of Software Development. It is something we definitely do not want but at the same time it is inevitable and cannot be deferred. The main idea behind was to accept this nature of Development and the team should be able to learn from them and make sure it never happens again. 

They showcased the vast Facebook's data centre all over the world. Each datacenter was built in order to manage massive traffic coming from not only Facebook but from Instagram, Whatsapp etc. They also showed the switches that they build for the data centres which they chose over purchasing from the vendors because of the cost. Each data centre was supposed to have a massive infrastructure to handle large traffic from the network. Now with such big infrastructure possibility of error is vast. Identifying and resolving them is another big challenge. They gave us an example of how one of the errors that occurred shut down the entire Facebook data centre. Performance is also important and with such large infrastructure, it becomes harder to optimize and increase performance. They showcased their software development methodology by giving an example of improving their performance of data transfer between networks to users.

Their methodology does not include agile planning or software development. It consists of abstract project development method which is as follows:
\begin{enumerate}
      \item Investigation
      \item Analysis
      \item Project Retrospect
      \begin{itemize}
            \item Set clear project goals
            \item If you cannot measure it, it did not happen
            \item Get the right stakeholders
            \item Requirements gathering
            \item Career progression is not linear
            \item Sunk cost fallacy (if not working move on)
      \end{itemize}
\end{enumerate}

The overall structure is quite similar to IBM's Vision, Plan and Develop. The words involved are quite different and so is their idea behind it. Usually, companies focus on software development as building a machine whereas Facebook's development is in terms of numbers if a project tends to increase the performance they ask themselves by how much or if it helps to solve some problem they ask themselves how many people would it help solve the problem. The number of people working in a team in Facebook is usually around 6-10. Most of the companies in my experience follow certain methodology like Agile, Waterfall etc. Whereas, Facebook leaves everything to the team. Facebook has more of an ``ownership'' way of development where they give the project to a team and then the project is owned by the team in a sense that they can decide whatever it seems fit.

Facebook has over billion users in each of their platforms so when it comes to errors, they need robust methodology to tackle the errors occurred and a plan so that it doesn't ever happen again. Their immediate step can be stated as:

\begin{itemize}
      \item Reboot servers
      \item Disable automation
      \item Check rest of the site's status
      \item look for config changes or deployment
\end{itemize}

These steps are simple enough to tackle errors occurred followed by small hack or fix to keep the site running for the time being. After this software engineers gather together in ``War room''. Where they try to tackle the problem keeping in mind the following guidelines:

\begin{itemize}
      \item Try to keep the rest of the site running
      \item Reproduce failures on a small scale
      \item Look for code/config changes
      \item Dig into and analyse data
\end{itemize}

Now, this is the immediate measures taken by the engineers at Facebook. But since we need something to fix the problem such so that it does not happen again. For this they try to:

\begin{itemize}
      \item Focus on the root cause (Deep dive and Not blame)
      \item Find how to prevent recurrence
      \item Do not rely on best intentions
      \item Always remember humans are desperately unreliable      
\end{itemize}

These are usually the guidelines and rules for tackling, handling and solving the errors that occur in the system. They showed with real-life errors that actually occurred on Facebook and how they followed the above steps and guidelines to fix the memory leak and overloading of traffic on servers problems.

To prevent failures unit testing, integration testing and load testing are one of the ways. But testing would not catch all the failure nodes and one can't test an entire distributed system at scale. Hence, they try to:
\begin{itemize}
      \item Roll out stuff slowly
      \item Monitor KPIs
      \item Fail fast than timeouts
      \item Defensive servers
\end{itemize}

One can see here that most of the points that are being showcased by the speakers are not hard and fast rules. They are also not in any book of best practices. These rules and regulations seem more like discoveries that they made and the method they found out to suit their needs using their past experiences. Learning and moving forward is one of the Facebook ideals and it can be seen in their methodology.

Since contribution, changes, code and bugs are all proportional to each other. Hence, small changes help them to figure out what broke when. The error tackling process is very different from the normal agile developments where most of the errors are caught before the development pipeline and if not immediate fix is created but proper fix is done in the later stages of the cycle of development whereas in Facebook they do not of have cycle of development, they release as they build and if error occurs they rollback and fix and then keep developing.

The code rolling out is in small chunks but every day. This is what surprised me! They not only roll out every day but several times a day. Whereas in agile, waterfall and kanban processes, rolling out of new code is every 2-4 weeks. Every team in a company usually follows a proper agile planning whereas in Facebook it is on ownership basis. Testing and code reviews are the norm these days and are followed by most of the companies even Facebook. But Facebook has a really large source code spread over several modules, computers, data centres. Hence running the test every time they deploy is almost impossible. Hence to tackle this they have tools that usually run against what they develop and only those test and their dependencies are run. Saving a lot of time in the process.

I feel such varied development process is due to the sheer number of computing and data. If Facebook had to use agile planning it would fail because rolling out such big change is risky. This risk comes because of its a distributed systems. Distribution is not only between computers but whole data centres. Rolling out to such large data centres is another impossible task. With such large code base testing, each of them would take ages. Finding error is another difficult task for them because of this sheer velocity and amount. Hence, they do something call reactive programming where they automate most of the tasks and wait for the signal from the system if something is wrong. Facebook also have automated tools for detecting software vulnerabilities, code smell etc. Facebook uses their own whereas IBM and other companies use SonarQube.

It is very surprising to see such different approach of development compared to IBM (which is discussed later). Their approach from development to testing is totally different, unique and interesting. Their approach is more of a summary of their solutions of their past mistakes. They not only understand that there is no best way to develop a particular software and every software has its own need. They also understand and realize the need for automating almost everything and using reactive programming where it's more about notification than polling for new information. Their understanding of these issues allows them to be so flexible.


\section{FoodCloud seminar}

\section{Speaker}

Roy Phillips, Chief Technical Officer at FoodCloud.

\section{Report}

The speaker started off with explaining about FoodCloud. FoodCloud has deals with the various grocery stores. When grocery stores cannot sell perfectly good food. They upload the description of the food using their in-store scanner or their smartphone app. FoodCloud has many links with the local charity. They let the local charity know about the unsold food that the grocery stores couldn't sell. The charity can respond to the notification by accepting and collecting it.

They not only have client and charity app, they have text notification service, a website and have various integration with their clients. The toughest part in FoodCloud is the integration of their platform with their clients i.e Tesco etc. We can see that IBM and Facebook mostly deal with providing services that they usually make from scratch and run on their own servers. Integration is usually one of their tasks but mostly they are the providers providing the services. FoodCloud, on the other hand, acts as a middleman where they create a bridge between the grocery stores and the charity. This requires software to be very versatile and flexible to be able to handle integration with ease.

Facebook did not seem to give much care to the software tools or platform they used. IBM mostly used Java for creating programs. FoodCloud uses Scala which addresses the criticisms of Java. Scala also uses Java VM. This is due to the robustness and cross-platform Java that the FoodCloud chose to use. FoodCloud has to integrate with multiple clients hence they do require cross-platform scalable tools to be used.

In regards to the running of the application, Facebook and IBM have their own big data centres as they are such large companies. Whereas since FoodClous is a growing non-profit business, they need a scalable solution. For this reason, FoodCloud chose Heroku for deploying the apps for its ease of deploying and scalable plus they also are expandable based on needs meaning they only have to pay for what they use.

Tracking of the progress of the software development was done using JIRA. They employed simple backlog mechanism where they pick the most important tasks and work on it. Tracking the task in the backlog. They followed Kanban and Agile methodology. The speaker did not believe in estimating the time required to finish the task. As in software development, it's very hard to estimate how much time it is gonna take to complete a task. Wasting time on estimation is hence unnecessary. This ideology is quite valid in real life. Spikes are very common in software development. Spike is essentially when trying to finish a task, the developers find that this task actually requires major work and more time than the estimation. These spikes tend to obstruct the agile process. Hence the speaker followed a methodology where his team took the maximum of 4 stories and worked on them. 

The speaker strongly followed the SOLID principles. SOLID principles are the design principles in OOPs. FoodCloud used technology and designs such as SRP, Akka Streams, RabbitMQ for messaging etc. Testing was given a lot of importance by the speaker. I feel the technologies chosen by the speaker are indeed quite robust and scalable. The hosting on Heroku does make sense as the want minimize their cost. They follow continuous integration and development. 

I feel that the process and the tools are well suited and chosen for the development of FoodCloud. It is interesting to see different methodologies and process used by various different companies. The number of projects, the size of the project and the type of the project drive the type and methodology chosen. IBM used one process to manage different teams namely agile methodology. Whereas Facebook did not care much about the process and gave the developers the freedom to choose their methodology. It is interesting that Roy takes a different approach quite similar to Facebook. Rather than following strict methodology they tend to focus on simple backlog like implementation. This is due to the sheer size of users and the size of the company. Another reason for choosing so that I believe is that FoodCloud requires them to interact with various developers of the client the company is working with to integrate their system.


\section{Comparison of Company Seminars}
This section compares the methodology and tools used by the 3 different companies namely IBM, Facebook and FoodCloud. All the companies gave a sneak peak of their development process and their preferences for various methodologies. IBM and FoodCloud take an agile approach to building applications where Facebook leaves it to the team to decide upon whatever they think fit for the particular task. What if IBM uses Facebook way of development it might not be as productive. The reason is IBM has way more employees than Facebook and managing and tracking would become a difficult task. Also, most of the Facebook's apps run on their servers, hence they provide services. Whereas IBM does both they provide services by running software on their machines and they also provide software to be run and used by the buyer. Hence rolling out would not be possible every now and then. Also, rigorous testing is required as rolling back or updating a software running on someone else's machine is difficult and inconvenient. So, for making sure that their rollout is stable their software has to take an agile process of development.

This can be said for Facebook as well if Facebook used Agile approach throughout. Facebook has a number of data centres and most of their apps are based on distributed computing. Since they as a company provide services rather than proper software. This distributed computing makes it difficult to roll out all new features in one go and also increases chances of error. Hence, they require rollout slowly and more frequently. It also harder to roll back and find the error causing in a distributed system. One of most important thing comparing IBM, FoodCloud against Facebook is that Facebook uses reactive programming and almost have all their task automated. The reason for this is the sheer number of computers in the data centres as well as the huge code base of Facebook. Too much automation actually not needs by FoodCloud as they have much smaller scale and also they are already use very advanced and tools and services like Heroku for developing and deploying. 

All the three companies agree to the importance of Refactoring and Testing. In IBM and FoodCloud when the developers believe that the new requirement coming cannot integrate with the current design of the system and requires refactoring. The developers add refactoring as a task in their backlog and start performing them in one of the sprints. In Facebook, its more relaxed refactoring is usually done by the individual as and when they feel like it needs to be done. Since there is no specific scheme throughout they are flexible and give the developers the option to do refactoring when and how they seem fit. Testing is the major aspect of development. Facebook focuses on Unit test a lot as the integration and parallel/distributed testing of Facebook is really slow and hard. IBM and FoodCloud focused on all types of testing. IBM had another layer of Testing known as UAT which was focussed more on than other companies. In this usually, the UAT testers will go through the site as a user and see if the new features are working correctly. This is almost impossible for Facebook as their applications are too large. To this Facebook actually has tools that test code and its dependencies that are changed.

IBM, Facebook and FoodCloud all create models of software before getting into code. But none of them actually focuses too much on the UML diagram or creating models for each and every codebase. They all preferred documenting the overall structure of the system rather than going too deep in the documentation. Since software has ever changing need. Documentation actually becomes old and useless faster in such iterative programming environment.

It is quite interesting to see that each company followed the path that they believe is best for current needs. And if one tries to interchange the system, one can see it is not as effective as their current one. They all created and used tools that are right for them at that point in time. But all of the companies are very flexible in the sense to accept change. Software development has an ever changing and ever growing environment and fixating on the tools and procedure does not let you far. As no model is right model. The right model is the model that works for current needs with giving a little thought to the future.