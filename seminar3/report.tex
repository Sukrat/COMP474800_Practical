\documentclass[10pt]{article}    
    \usepackage[utf8]{inputenc}
    \usepackage[margin=3cm]{geometry}
    
    \title{\vspace{-3.0cm}COMP47480 Seminar 3: FoodCloud}
    \author{Sukrat Kashyap (14200092)}
    \date{10 April 2018}
    \setlength{\parindent}{0cm}
    \setlength{\parskip}{0.7em}
    
\begin{document}

\maketitle

\section{Speaker}

Roy Phillips, Chief Technical Officer at FoodCloud.

\section{Report}

The speaker started off with explaining about FoodCloud. FoodCloud has deals with the various grocery stores. When grocery stores cannot sell perfectly good food. They upload description of the food using their in-store scanner or their smartphone app. FoodCloud has many links with the local charity. They let the local charity know about the unsold food that the grocery stores couldn't sell. The charity can respond to the notification by accepting and collecting it.

They not only have client and charity app, they have text notification service, a website and have various integration with their clients. The toughest part in FoodCloud is integration of their platform with the their clients i.e Tesco etc. We can see that IBM and Facebook mostly deals with providing services that they usually make from scratch and run on their own servers. Integration is usually one of their task but mostly they are the providers providing the services. FoodCloud on the other hand acts as a middle man where they create a bridge between the grocery stores and the charity. This requires software to be very versatile and flexible to be able to handle integration with ease.

Facebook did not seem to give much care to the software tools or platform they used. IBM mostly used Java for creating programs. FoodCloud uses Scala which addresses the criticisms of Java. Scala also uses Java VM. This is due to the robustness and cross-platform Java that the FoodCloud chose to use. FoodCloud has to integrate with multiple clients hence they do require cross-platform scalable tools to be used.

In regards to the running of the application, Facebook and IBM have their own big data centres as they are such large companies. Whereas since FoodClous is a growing non-profit business, they need a scalable soution. For this reason, FoodCloud chose Heroku for deploying the apps for its ease of deploying and scalable plus they also are expandable based on needs meaning they only have to pay for what they use.

Tracking of the progress of the software development was done using JIRA. They employed simple backlog mechanismm where they pick the most important tasks and work on it. Tracking the task in the backlog. They followed Kanban and Agile methodology. The speaker did not believe in estimating the time required to finish the task. As in software development its very hard to estimate how much time it is gonna take to complete a task. Wasting time on estimation is hence unnecessary. This ideology is quite valid in real life. Spikes are very common in software development. Spike is essentially when trying to finish a task, the developers find that this task actually requires major work and more time than the estimation. This spikes tend to obtruct the agile process. Hence the speaker followed a methodology where his team took maximum of 4 stories and worked on them. 

The speaker strongly followed the SOLID principles. SOLID principles are the design principles in OOPs. FoodCloud used technology and designs such as SRP, Akka Streams, RabbitMQ for messaging etc. Testing was given a lot of importance by the speaker. I feel the technologies chosen by the speaker are indeed quite robust and scalable. The hosting on Heroku does make sense as the want minimize their cost. They follow continuous integration and development. 

\section{Q \& A}

Some question that were asked at the end of seminar were:

\begin{enumerate}
    \item Q\\
        Answer: 
\end{enumerate}

\end{document}
